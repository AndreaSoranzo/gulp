\section{Introduzione}
\subsection{Scopo del documento}
Il presente documento fornisce una guida dettagliata e organizzata per la
pianificazione, l'esecuzione e il controllo delle attività coinvolte nello
sviluppo del prodotto. Esso definisce gli obiettivi del progetto, specifica i
risultati attesi, identifica le risorse necessarie e i budget corrispondenti
attraverso stime preventive. Il documento è pertanto strutturato come segue:
\begin{itemize}
      \item \textbf{Analisi dei rischi:} identifica i potenziali rischi individuati
            dal gruppo, la probabilità che si verifichino, la loro gravità e le relative
            strategie di mitigazione;
      \item \textbf{Modello di sviluppo:} descrive il modello di sviluppo adottato dal
            gruppo, da cui derivano vincoli sulla pianificazione e gestione di progetto;
      \item \textbf{Pianificazione:} offre una visione dettagliata del piano di lavoro
            del gruppo, basata su una chiara suddivisione delle attività. La pianificazione
            è strutturata in sprint, ognuno dei quali rappresenta un'unità di tempo
            incentrata su obiettivi specifici;
      \item \textbf{Preventivo:} indica le ore e i costi che si intende impiegare
            in ogni periodo pianificato;
      \item \textbf{Consuntivo:} indica le ore e i costi effettivamente impiegati
            in ogni periodo pianificato, mettendoli a confronto con i dati preventivati;
      \item \textbf{Attualizzazione dei rischi:} descrive i rischi che si sono verificati
            durante lo svolgimento del progetto e le misure adottate conseguentemente.
\end{itemize}

\subsection{Scopo del prodotto}
La data visualization è il processo di traduzione di dati in grafici e altri
elementi visivi, al fine di sfruttare la facilità di elaborare le immagini da
parte del nostro cervello, agevolando così il processo decisionale. Di fatto, è
anche una parte enorme del processo di analisi dei dati e un modo efficiente
per comunicare le informazioni in modo universale, veloce ed efficace per
tutti. Per questo il \textit{Proponente} Sanmarco Informatica S.p.A. richiede
di sviluppare un'interfaccia web per la visualizzazione in forma
tridimensionale di dati provenienti da diverse fonti (\textit{database},
\textit{REST API}, \dots) tramite istogrammi 3D che siano navigabili e
interattivi. Inoltre gli stessi dati devono essere visualizzati anche in forma
tabellare.
\subsection{Glossario}
Al fine di evitare incomprensioni o ambiguità relativamente alla terminologia
usata all'interno del documento, viene fornito un Glossario nel file
\textit{Glossario V1.0.0} in grado di dare una definizione precisa per i
termini potenzialmente ambigui. Tali termini saranno facilmente identificabili
grazie a una lettera G a pedice.
\subsection{Riferimenti normativi}
\begin{itemize}
      \item \textit{Norme di progetto} \\ \url{https://6bitbusters.github.io/norme_di_progetto.pdf}
      \item Capitolato d'appalto C5 - Sanmarco Informatica S.p.A.: 3Dataviz \\ \url{https://www.math.unipd.it/~tullio/IS-1/2024/Progetto/C5.pdf}
      \item Regolamento del progetto didattico \\ \url{https://www.math.unipd.it/~tullio/IS-1/2024/Dispense/PD1.pdf}
\end{itemize}
\subsection{Riferimenti informativi}
\begin{itemize}
      \item Slide T2 - Corso di Ingegneria del Software - I processi di ciclo di vita del \textit{software} \\ \url{https://www.math.unipd.it/~tullio/IS-1/2024/Dispense/T02.pdf}
      \item Slide T3 - Corso di Ingegneria del Software - Modelli di sviluppo \textit{software} \\ \url{https://www.math.unipd.it/~tullio/IS-1/2024/Dispense/T03.pdf}
      \item Slide T4 - Corso di Ingegneria del Software - Gestione di progetto \\ \url{https://www.math.unipd.it/~tullio/IS-1/2024/Dispense/T04.pdf}
\end{itemize}
